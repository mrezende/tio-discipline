% slides-exemplo-beamer
%
% Sb Out  1 19:47:43 BRT 2011

\PassOptionsToPackage{hyphens}{url}
\documentclass[table, usenames, svgnames, dvipsnames, hyperref={pdftex,plainpages=false,pdfpagelabels,pagebackref,colorlinks=true,citecolor=DarkGreen,linkcolor=NavyBlue,urlcolor=DarkRed,filecolor=green,bookmarksopen=true}]{beamer}
\usepackage{beamerthemeshadow}
\usepackage[T1]{fontenc}
\usepackage[brazilian]{babel}
\usepackage[latin1]{inputenc}
\usepackage[absolute,overlay]{textpos}
\usepackage{amssymb}
\usepackage{array}

\usepackage[fixlanguage]{babelbib}

%\PassOptionsToPackage{hyphens}{url}\usepackage[pdftex,plainpages=false,pdfpagelabels,pagebackref,colorlinks=true,citecolor=DarkGreen,linkcolor=NavyBlue,urlcolor=DarkRed,filecolor=green,bookmarksopen=true]{hyperref} % links coloridos

%\usepackage[all]{hypcap}                % soluciona o problema com o hyperref e capitulos
\usepackage[square,sort,nonamebreak,comma]{natbib}  % cita��o bibliogr�fica alpha (alpha-ime.bst)
\usepackage{setspace}                   % espa�amento flex�vel

\graphicspath{{./images/}}

\linespread{1.05} % Change line spacing here, Palatino benefits from a slight increase by default

\makeatletter
\renewcommand\@biblabel[1]{\textbf{#1.}} % Change the square brackets for each bibliography item from '[1]' to '1.'
\renewcommand{\@listI}{\itemsep=0pt} % Reduce the space between items in the itemize and enumerate environments and the bibliography

%\usepackage{subfigure}
%\usepackage{multicol}
%\usepackage{colortbl}

% ---------------------------------------------------------------------------- %
% Defini��es beamer
% ---------------------------------------------------------------------------- %
\usetheme{Rochester}
%\usetheme{Luebeck}
\usecolortheme{rose}

% Algumas defini��es para o layout
\setbeamerfont{frametitle}{size=\normalsize}
\setbeamerfont{title}{size=\normalsize}
\beamertemplatenavigationsymbolsempty

% Podem ser utilizadas imagens no background
%\setbeamercolor{frametitle}{bg=black}
%\usebackgroundtemplate{\includegraphics[height=\paperheight]{figuras/back.jpg}}

% Os simbolos de navegacao nao sao necessarios
\setbeamertemplate{navigation symbols}{}
\setbeamertemplate{footline}{}

%\setbeamertemplate{footline}[page number]{}
%\setbeamertemplate{footline}[text line]{ \hfill {\insertframenumber}}

% Indice para cada seo (aparece antes de cada section)
\AtBeginSection[]
{
	\begin{frame}<handout:0>
		\frametitle{\textbf{Agenda}}
		\footnotesize{ \tableofcontents[currentsection,hideothersubsections] }
	\end{frame}
}

% ---------------------------------------------------------------------------- %
% Declara��es
% ---------------------------------------------------------------------------- %
\DeclareGraphicsExtensions{.pdf,.jpg,.png} % compilamos apenas com pdflatex



% semitransparente
\newcommand{\semitransp}[2][35]{\color{fg!#1}#2}

% \definecolor{myred}{rgb}{0.8, 0.3, 0.3}
\definecolor{myblue}{rgb}{0.2, 0.2, 0.70196}

\usepackage{framed} % utilizado para codigo fonte
\definecolor{shadecolor}{named}{LightGray}

% ---------------------------------------------------------------------------- %
% T�tulo
% ---------------------------------------------------------------------------- %
\title{\textbf{Tecnologias de Informa��o e Organiza��es}}

\author[Marcelo de Rezende Martins]{\scriptsize
    Marcelo de Rezende Martins\\
    rezende.martins@gmail.com
}

\subtitle{}

\institute{\\[1.0mm]
Instituto de Pesquisas Tecnol�gicas\\
IPT}

\date{{\tiny \today}}


% ---------------------------------------------------------------------------- %
\begin{document}
% ---------------------------------------------------------------------------- %

% ---------------------------------------------------------------------------- %
% Primeira p�gina: slide 0
% ---------------------------------------------------------------------------- %

{%\usebackgroundtemplate{}}
\begin{frame}[plain]


	\titlepage

	\addtocounter{framenumber}{-1}
\end{frame}
}


% ---------------------------------------------------------------------------- %
% slide 1 >
% ---------------------------------------------------------------------------- %
%\setbeamertemplate{footline}{\hrule \MyLogo } %\hfill\includegraphics[height=1.2cm]{figuras/olho1.png}}
%\setbeamertemplate{footline}{\hfill\includegraphics[height=1.2cm]{figuras/olho1.png}}
\setbeamertemplate{navigation symbols}{\large {\insertframenumber}}

% ---------------------------------------------------------------------------- %

\begin{frame}[plain]
\frametitle{}
	\begin{block}{Artigos}
		\begin{enumerate}
			\item What Happens If Net Neutrality Goes Away? \cite{orcutt:17}
			\item The relentless push to add connectivity to home gadgets is creating dangerous side effects that figure to get even worse. \cite{schneier:17}
			\item The Best and Worst Internet Experience in the World \cite{larson:16}
			\item How Ad Blockers Have Triggered an Arms Race on the Web \cite{arxiv:16}
	\end{enumerate}
	\end{block}
\end{frame}

% ---------------------------------------------------------------------------- %
\begin{frame}
\frametitle{\textbf{Agenda}}

	\hspace*{+4.0em}
	\footnotesize{ \tableofcontents }
\end{frame}


% ---------------------------------------------------------------------------- %
\section{Vis�o Geral}
% ---------------------------------------------------------------------------- %

\begin{frame}
\frametitle{Vis�o geral dos artigos}

	\begin{enumerate}
			\item Neutralidade da rede
			\item Seguran�a na IoT
			\item Censura na internet
			\item Ad Blockers
	\end{enumerate}

\end{frame}


% ---------------------------------------------------------------------------- %
\section{Neutralidade da rede}
% ---------------------------------------------------------------------------- %

% ---------------------------------------------------------------------------- %
\subsection{Inova��o e neutralidade da rede}
% ---------------------------------------------------------------------------- %

\begin{frame}
\frametitle{\textbf{Neutralidade da rede}}

	� ben�fico ou prejudicial � inova��o? \cite{nordrum:17}
	
\end{frame}

% ---------------------------------------------------------------------------- %
\section{Privacidade}
% ---------------------------------------------------------------------------- %

\begin{frame}
\frametitle{Paradoxo da privacidade}
	O que � privacidade na internet? \cite{notetoself:17, eff:17, effdnt:17, doi:10.1080/1369118X.2013.777757, nprprivacy:17, propublicafb:16}
\end{frame}

% ---------------------------------------------------------------------------- %
{%\usebackgroundtemplate{}}
\begin{frame}[plain]

	\titlepage

	\addtocounter{framenumber}{-1}
\end{frame}
}

%----------------------------------------------------------------------------------------
%	BIBLIOGRAPHY
%----------------------------------------------------------------------------------------
\begin{frame}[allowframebreaks]
\frametitle{Refer�ncias}
\singlespacing   % espa�amento simples
\bibliographystyle{alpha-ime}% cita��o bibliogr�fica alpha
\bibliography{references}  % associado ao arquivo: 'bibliografia.bib'
\end{frame}

\end{document}
